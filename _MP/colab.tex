PNLD0001-MP.tex:\SideImage{Lélia Gonzalez foi uma antropóloga e filósofa com estudos sobre a linguagem coloquial e o racismo na sociedade brasileira. (Cezar Loureiro, Revista Cult; Domínio público)}{PNLD0001-05.png}
PNLD0001-MP.tex-Por que o autor escreveu a ``carta'' de modo a apresentá"-la com os
--
PNLD0001-MP.tex:\Image{A quermesse retratada na foto traz paralelos com o «mafuá» e a feira de prendas relatado pelo autor. (Escola Estadual Antonio Vicente Azambuja; CC-BY-SA-4.0)}{PNLD0001-06.png}
PNLD0001-MP.tex-
--

--
PNLD0006-MP.tex:\SideImage{Excalibur, a espada da lenda do rei Arthur - ilustração feita em 1902 por Howard Pyle. (Howard Pyle; Domínio Público)}{PNLD0006-07.png}
PNLD0006-MP.tex-
--
PNLD0006-MP.tex:\Image{Manuscrito em que há uma representação do Rei Arthur, feito por Geoffrey of Monmout (1095 – 1155?), intitulado History of Kings of Britain - História dos Reis Britânicos. (Geoffrey of Monmouth; CC0)}{PNLD0006-04.png}
PNLD0006-MP.tex-
--
PNLD0006-MP.tex:\Image{Outro manuscrito medieval em que há uma representação do Rei Arthur. Ele segura sua espada e um brasão com a figura da Virgem Maria e seu filho. (British Library; Domínio Público)}{PNLD0006-05.png}
PNLD0006-MP.tex-
--
PNLD0006-MP.tex:\Image{A primeira sede dos cavaleiros templários, a Mesquita de Al-Aqsa, em Jerusalém, o monte do Templo. Os cruzados chamaram-lhe de o Templo de Salomão, como ele foi construído em cima das ruínas do templo original, e foi a partir desse local que os cavaleiros tomaram seu nome de templários. (Berthold Werner; CC-BY-SA 3.0)}{PNLD0006-10.png}
PNLD0006-MP.tex-
--
PNLD0006-MP.tex:\Image{Cavaleiros sentados ao redor da Távola Redonda (Biblioteca Nacional da França; Domínio Público)}{PNLD0006-06.png}
PNLD0006-MP.tex-
--
PNLD0006-MP.tex:\SideImage{Quadro intitulado ``O Santo Graal'', de Dante Gabriel Rossetti (1860) (Dante Gabriel Rossetti; Domínio Público)}{PNLD0006-11.png}
PNLD0006-MP.tex-
--
PNLD0006-MP.tex:\SideImage{Poeta Zé da Luz (1904-1965), à direita da foto. Fonte: Wikipedia. CC-BY.}{PNLD0006-20.jpg}
PNLD0006-MP.tex-
--
PNLD0007-MP.tex:\SideImage{Paul D. Escott, autor do prefácio. (Wake Forest University; Acervo pessoal do autor)}{PNLD0007-03}
PNLD0007-MP.tex-
--
PNLD0007-MP.tex:\SideImage{Nina Simone foi uma artista que não separou a arte de seu papel político contestatório na sociedade racista de seu país, os \textsc{eua}. (Gerrit de Bruin ; CC BY 4.0)}{PNLD0007-18}
PNLD0007-MP.tex-
--
PNLD0007-MP.tex:\SideImage{Luedji Luna busca em seu trabalho reconstruir os laços destruídos pelo sistema escravagista. (Marcelo Nava; CC BY-NC-ND 2.0)}{PNLD0007-17}
PNLD0007-MP.tex-
--
PNLD0007-MP.tex:\Image{Fuga em massa de escravos de Cambridge, Maryland, \textsc{eua}, 1857. (William Still; CC-BY-NC 2.0)}{PNLD0007-04}
PNLD0007-MP.tex-
--
PNLD0007-MP.tex:\Image{Carolina Maria de Jesus foi uma escritora brasileira conhecida pelo livro ``Quarto de despejo'' onde narra sua vida num morro. (Arquivo Nacional; Domínio Público.)}{PNLD0007-21}
PNLD0007-MP.tex-
--
PNLD0007-MP.tex:\SideImage{Frantz Fanon foi um dos primeiros a estudar os efeitos 
PNLD0007-MP.tex-psíquicos do sistema colonial na população negra. 
--
PNLD0007-MP.tex: \Image{Nota publicada no Diário do Brazil em 1888.``A desconfiança é geral. O capital se retrahe. O espirito de empreza desaparece''(Acervo digital da Biblioteca Nacional ; Domínio Público)}{PNLD0007-20}
PNLD0007-MP.tex-
--
PNLD0007-MP.tex:%\Image{Luiz Gama, conhecido como ``Libertador de Escravos'', foi ele mesmo escravizado e, quando criança, foi vendido pelo próprio pai para o tráfico. (Domínio Público)}{PNLD0007-22.png}
PNLD0007-MP.tex-
--
PNLD0007-MP.tex:  \Image{``Incômodo'', 2014. Sidney do Amaral. (Obra do acervo da Pinacoteca do Estado de São Paulo; Divulgação)}{PNLD0007-15}
PNLD0007-MP.tex-
--
PNLD0007-MP.tex:  \Image{``A libertação dos escravos'', 1889. Pedro Américo. (Acervo do Palácio dos Bandeirantes, Domínio Público)}{PNLD0007-16}
PNLD0007-MP.tex-
--
PNLD0007-MP.tex:\Image{Geo Simmons, um dos entrevistados. (Library of Congress; Domínio Público)}{PNLD0007-11.png}
PNLD0007-MP.tex-
--
PNLD0007-MP.tex:\Image{Clara Brim, uma das entrevistadas. (Library of Congress; Domínio Público)}{PNLD0007-12.png}
PNLD0007-MP.tex-
--
PNLD0007-MP.tex:\SideImage{Sol Waltson, um dos entrevistados. (Library of Congress; Domínio Público)}{PNLD0007-13.png}
PNLD0007-MP.tex-
--
PNLD0007-MP.tex:\SideImage{Ellen Butler, uma das entrevistadas. (Library of Congress; Domínio Público)}{PNLD0007-14.png}
PNLD0007-MP.tex-
--
PNLD0007-MP.tex:\SideImage{Em um campo de algodão na Carolina do Sul, \textsc{eua}. (Okinawa Soba/Flickr; CC-BY-NC 2.0)}{PNLD0007-09.png}
PNLD0007-MP.tex-
--
PNLD0007-MP.tex:\Image{Antigo mercado de escravos, Flórida, \textsc{eua}. (Okinawa Soba/Flickr; CC-BY-NC 2.0)}{PNLD0007-10.png}
PNLD0007-MP.tex-
--
PNLD0008-MP.tex:\SideImage{O autor da obra, Fabio Atui. (Acervo pessoal)}{PNLD0008-03.png}
PNLD0008-MP.tex- 
--
PNLD0008-MP.tex:\Image{Bacia Amazônica com a localização do rio Negro (Kmusser; CC-BY-SA 3.0)}{PNLD0008-05.png}
PNLD0008-MP.tex-
--
PNLD0008-MP.tex:\Image{Floresta Amazônica, vista do alto (Lúcia Barreiros; CC-BY 3.0 br)}{PNLD0008-10.png}
PNLD0008-MP.tex-
--
PNLD0008-MP.tex:\Image{Vista aérea da comunidade de Assunção do Içana, no município de São Gabriel da Cachoeira (AM), na região do Alto do Rio Negro (Marcelo Camargo/Agência Brasil; CC-BY 3.0 br)}{PNLD0008-09.png}
PNLD0008-MP.tex-
--
PNLD0008-MP.tex:\Image{Praia no Alto Rio Negro, uma das regiões em que estão os aprox. 19 060 falantes de Nheengatu, de onde se origina o termo ``Paranã Tipi''. (Isabelle Allet-Coche ; CC-BY-SA 4.0)}{PNLD0008-04.png}
PNLD0008-MP.tex-
--
PNLD0008-MP.tex:\Image{Localização de São Gabriel da Cachoeira, cidade em que 74\% dos habitantes são indígenas. As línguas oficiais no município, ao lado do português, são o nheengatu, o  tucano e o baníua. (Marcos Elias de Oliveira Júnior ; CC-BY 2.5)}{PNLD0008-06.png}
PNLD0008-MP.tex-
--
PNLD0008-MP.tex:\Image{O Festival Cultural dos Povos Indígenas do Alto Rio Negro/AM, 
PNLD0008-MP.tex-realizado anualmente em São Gabriel da Cachoeira. (Hans Denis Schneider; CC-BY-NC-SA)}{PNLD0008-07.png}
--
PNLD0008-MP.tex:\Image{Apresentação das agremiações no Festival Cultural dos Povos Indígenas do Alto Rio Negro/AM. (Hans Denis Schneider; CC-BY-NC-SA)}{PNLD0008-08.png}
PNLD0008-MP.tex-
--
PNLD0009-MP.tex:\Image{Ryūnosuke Akutagawa, depois de graduado em Literatura Inglesa, torna"-se
PNLD0009-MP.tex-discípulo do escritor japonês Natsume Soseki (Ogawa Kazumasa; Domínio
--
PNLD0009-MP.tex:\SideImage{Cartaz original do filme de Akira Kurosawa (Flickr, CC BY 2.0)}{PNLD0009-20.png} 
PNLD0009-MP.tex-
--
PNLD0009-MP.tex:\Image{Samurai aprisionado por bandido. <<Rashômon>> (1050), de A.
PNLD0009-MP.tex-Kurosawa (\textit{Still} do filme; Divulgação)}{PNLD0009-21.png} 
--
PNLD0009-MP.tex:\Image{Filme de Akira Kurosawa baseado em dois contos de Ryūnosuke Akutagawa:
PNLD0009-MP.tex-``Rashômon'' e ``Dentro do bosque'' (Daiei; Domínio Público)}{PNLD0009-06.png}
--
PNLD0009-MP.tex:\Image{Noiva do Samurai em perigo. <<Rashômon>>
PNLD0009-MP.tex-(1050), de A. Kurosawa (\textit{Still} do filme; Divulgação)}{PNLD0009-22.png} 
--
PNLD0009-MP.tex:\Image{Noiva ameaçada. <<Rashômon>> (1050), de A.
PNLD0009-MP.tex-Kurosawa (\textit{Still} do filme; Divulgação)}{PNLD0009-23.png}
--
PNLD0009-MP.tex:\Image{Imagem do personagem Akutagawa invocando seu portal Rashômon, 
PNLD0009-MP.tex-no mangá Bungo Stray Dogs. Fonte: 
--
PNLD0009-MP.tex:%\Image{"A viagem de Chihiro" (2001), Studio Ghibli.  (Srchaos; CC-BY-SA 4.0)}{PNLD0009-10.png}
PNLD0009-MP.tex-
--
PNLD0009-MP.tex:\Image{Morador de vilarejo
PNLD0009-MP.tex-relembrando a trágica história. <<Rashômon>> (1050), de A.
--
PNLD0009-MP.tex:%\Image{Painel da Era Meiji  (Khalili Collections; CC-BY-SA
PNLD0009-MP.tex-%4.0)}{PNLD0009-05.png}
--
PNLD0009-MP.tex:\Image{Mulher japonesa do período Meiji montada em um cavalo de carga,
PNLD0009-MP.tex-1892--1895 (T. Enami ; Domínio Público)}{PNLD0009-09.png}
--
PNLD0009-MP.tex:\Image{<<Leg of the horse>>, manuscrito do autor, 1925, Museu Edo"-Tokyo -– Sumida,
PNLD0009-MP.tex-Tóquio, Japão. (Edo-Tokyo Museum; Domínio Público)}{PNLD0009-08.png}
--
PNLD0009-MP.tex:\Image{Foto em grupo com Kan Kikuchi, Ryūnosuke Akutagawa, Mutō Chōzō, Nagami
PNLD0009-MP.tex-Tokutarō (da esquerda para direita), 1919. (Sem Autoria; Domínio
--
PNLD0010-MP.tex:\Image{Félix Nadar (1920-1910), notável fotógrafo francês. Foi também 
PNLD0010-MP.tex-jornalista e escritor.}{PNLD0010-20}
--
PNLD0010-MP.tex:\Image{Jantar na Villa Kyrial, Vila Mariana, São Paulo.}{PNLD0010-21}
PNLD0010-MP.tex-
--
PNLD0010-MP.tex:\Image{O novo traçado das cidades atravessou por completo o plano original 
PNLD0010-MP.tex-da capital francesa reestruturando a cidade interligada por parques e jardins.}{PNLD0010-24}
--
PNLD0010-MP.tex:\SideImage{Barão de Haussmann, urbanista responsável pela reconstrução da Paris moderna}{PNLD0010-25}
PNLD0010-MP.tex-
--
PNLD0010-MP.tex:% \Image{As avenidas e bulevares de Paris foram uma novidade do novo plano urbanístico para a cidade, mas também uma forma de controlar o fluxo das manifestaões e impedir barricadas.}{PNLD0010-23}
PNLD0010-MP.tex-
--
PNLD0010-MP.tex:\Image{Charles Baudelaire em 1863. (Etienne Carjat; CC-BY 2.0)}{PNLD0010-03.png}
PNLD0010-MP.tex-
--
PNLD0010-MP.tex:\Image{Retrato de Baudelaire pintado por Émile Deroy, em 1844. (Émile Deroy; CC-BY-SA-4.0)}{PNLD0010-05.png}
PNLD0010-MP.tex- 
--
PNLD0010-MP.tex:%\Image{Desenho de Baudelaire feito pelo artista Édouard Manet (Édouard Manet; Domínio Público)}{PNLD0010-06.png}
PNLD0010-MP.tex-
--
PNLD0010-MP.tex:\Image{O escritor Edgar Allan Poe, grande inspiração de Baudelaire. (Autor desconhecido, fotografia restaurada por Yann Forget e Adam Cuerden ; Domínio Público)}{PNLD0010-08.png}
PNLD0010-MP.tex-
--
PNLD0010-MP.tex:\Image{Estátua ``O Flaneur": um tipo social retratado pelo autor, que representa o caminhante de galerias de grandes cidades. (\textsc{pvt} Pauline; CC-BY-SA 3.0)}{PNLD0010-07.png}
PNLD0010-MP.tex-
--
PNLD0010-MP.tex:\Image{``Carnaval parisiense": a sensação de estar imerso na multidão na cidade é um dos assuntos tratados pelo autor.  (Anônimo; CC-BY-SA 3.0)}{PNLD0010-10.png}
PNLD0010-MP.tex-
--
PNLD0010-MP.tex:\Image{A atmosfera de Paris é retratada pelo autor, especialmente no tema da solidão. (Stockholm Transport Museum; Domínio Público)}{PNLD0010-09.png}
PNLD0010-MP.tex-
--
PNLD0011-MP.tex:\Image{Bram Stoker, 1906 (National Portrait Gallery, Londres; Domínio Público)}{PNLD0011-03.png}
PNLD0011-MP.tex-
--
PNLD0011-MP.tex:\SideImage{Alvares de Azevedo introduziu a Literatura Gótica no Brasil (Acervo Digital da Biblioteca Nacional; Domínio Público)}{PNLD0011-12.png}
PNLD0011-MP.tex-
PNLD0011-MP.tex:\SideImage{Augusto dos Anjos, representante da Literatura Gótica (Acervo Digital da Biblioteca Nacional; Domínio Público)}{PNLD0011-13.png}
PNLD0011-MP.tex-
--
PNLD0011-MP.tex:\Image{Alphonsus Guimaraens também representava a Literatura Gótica (Acervo Digital da Biblioteca Nacional; Domínio Público)}{PNLD0011-14.png}
PNLD0011-MP.tex-
--
PNLD0011-MP.tex:\SideImage{Primeira edição alemã do livro do autor, "Drácula", Max Altmann, Leipzig 1908. (Selfie756; CC-BY-SA-4.1)}{PNLD0011-05.png}
PNLD0011-MP.tex-
--
PNLD0011-MP.tex:\Image{Em Whitby, Yorkshire,  no Reino Unido, a Abadia de Whitby foi uma inspiração para Bram Stoker em "Drácula". (Michael D Beckwith; CC0)}{PNLD0011-06.png}
PNLD0011-MP.tex-
--
PNLD0011-MP.tex:\Image{O castelo é um importante monumento nacional romeno e serviu como inspiração ao autor para a criação de "Drácula". (Pixabay; Domínio Público)}{PNLD0011-07.png}
PNLD0011-MP.tex-
--
PNLD0011-MP.tex:\Image{A obra \emph{Carmilla} de Joseph Sheridan Le Fanu, de 1872, possivelmente inspirou a temática e o gênero literário de \emph{Drácula} (David Henry Friston; Domínio Público)}{PNLD0011-08.png}
PNLD0011-MP.tex-
PNLD0011-MP.tex:%\Image{A obra "The Vampyre", de 1819, também teria inspirado o autor em seu livro "Drácula" (Houghton Library; Domínio Público)}{PNLD0011-09.png}
PNLD0011-MP.tex-
--
PNLD0011-MP.tex:\Image{A obra \emph{The Vampyre}, de 1819, também teria inspirado o autor em seu livro \emph{Drácula} (Houghton Library; Domínio Público)}{PNLD0011-09.png}
PNLD0011-MP.tex-
--
PNLD0011-MP.tex:\SideImage{Local de nascimento de Bram Stoker, 15 Marino Crescent, Dublin. (Smirkybec; CC-BY-SA-4.0)}{PNLD0011-04.png}
PNLD0011-MP.tex-
--
PNLD0011-MP.tex:\Image{Retrato de Vlad Tepes (governou 1455-1462, 1483-1496) (Utopia, Univeridade do Texas; Domínio Público)}{PNLD0011-11.png}
PNLD0011-MP.tex-
--
PNLD0013-MP.tex:\SideImage{Pintura de Daniel Defoe por Godfrey Kneller (National Maritime Museum; Domínio Público)}{PNLD0013-03.png}
PNLD0013-MP.tex-
--
PNLD0013-MP.tex:\Image{Robinson Crusoé parte e seu barco naufraga (Alexander Frank Lydon (A. F. Lydon); Domínio Público)}{PNLD0013-10.png}
PNLD0013-MP.tex-
--
PNLD0013-MP.tex:\Image{Robert Paton Gibbs, como Robinson Crusoé, no filme mudo americano "Robinson Crusoé" de 1916. (Henry W. Savage Prod.; Domínio Público)}{PNLD0013-07.png}
PNLD0013-MP.tex-
--
PNLD0013-MP.tex:\Image{Rota de Robinson Crusoé pelas montanhas dos Pirenéus (Alan Mattingly; CC-BY-SA 4.0)}{PNLD0013-06.png}
PNLD0013-MP.tex-
--
PNLD0013-MP.tex:\Image{Na legenda da ilustração: ``Robinson Crusoé e Sexta-feira atacando locais'' (Alexander Frank Lydon (A. F. Lydon); Domínio Público)}{PNLD0013-09.png}
PNLD0013-MP.tex-
--
PNLD0013-MP.tex:\SideImage{Fotografia da Escultura de Alexander Selkirk, personagem que inspirou o livro \emph{Robinson Crusoé}, 2004. (Felipe Henríquez González; CC-BY-SA 3.0)}{PNLD0013-05.png}
PNLD0013-MP.tex-
--
PNLD0013-MP.tex:\Image{Mapa xilografado e aquarelado do mundo antigo, 1513 (Arquivo Digital da Biblioteca Nacional; Domínio Público)}{PNLD0013-04.png}
PNLD0013-MP.tex-
--
PNLD0013-MP.tex:\Image{Harry Myers no filme ``As Aventuras de Robinson Crusoé'' (1922) (Universal Film Manufacturing Company/ Internet Archive; Domínio Público)}{PNLD0013-08.png}
PNLD0013-MP.tex-
--
PNLD0014-MP.tex:\Image{Busto de Maria Firmina dos Reis, localizado na Praça do Pantheon, em São Luís (MA) (Ramsessantos; CC-BY-SA-4.0)}{PNLD0014-03.png}
PNLD0014-MP.tex-
--
PNLD0014-MP.tex:\SideImage{Um dos mais importantes escritores na literatura do Ocidente é negro: Machado de Assis. (Autor desconhecido; Creative Commons)}{PNLD0014-04.png}
PNLD0014-MP.tex-
--
PNLD0014-MP.tex:\SideImage{O quadro \emph{A música e a poesia}, do italiano Adolfo Wildt, de 1920. (Adolfo Wildt; CC-BY-SA-4.0)}{PNLD0014-05.png}
PNLD0014-MP.tex-
--
PNLD0014-MP.tex:\Image{A cantora e compositora paraibana Socorro Lira lançou, em 2019, o álbum \emph{Cantos à beira"-mar}, com músicas feitas a partir dos poemas do livro. (Daniel Kersys; CC-BY-SA-4.1)}{PNLD0014-06.png}
PNLD0014-MP.tex-
--
PNLD0014-MP.tex:\SideImage{Estátua de Zumbi dos Palmares, líder do quilombo dos Palmares, um exemplo de resistência dos escravizados. (Elza Fiúza/ABr -- Agência Brasil; CC-BY-3.0-BR)}{PNLD0014-08.png}
PNLD0014-MP.tex-
--
PNLD0014-MP.tex:\Image{Busto em homenagem a Luiz Gama, figura muito importante na luta da abolição da escravatura. (Everton Zanella Alvarenga; CC-0)}{PNLD0014-09.png}
PNLD0014-MP.tex-
--
PNLD0014-MP.tex:\SideImage{Foto do desfile da Escola Primeira da Mangueira, em 2019 (Vanessa Garcia; CC-BY 2.0)}{PNLD0014-10.png}
PNLD0014-MP.tex-
--
PNLD0014-MP.tex:\Image{Quadro \emph{Abolição da Escravatura} de Victor Meirelles, de 1888 (Victor Meirelles; Brasiliana Iconográfica)}{PNLD0014-07.png}
PNLD0014-MP.tex-
--
PNLD0014-MP.tex:\Image{Quadro "Navio Negreiro" de Johann Moritz Rugendas (1830) (Johann Moritz Rugendas; Domínio Público)}{PNLD0014-11.png}
PNLD0014-MP.tex-
--
PNLD0014-MP.tex:\Image{Quadro "O Jantar" de Jean Baptiste Debret (1839) (; Domínio Público)}{PNLD0014-12.png}
PNLD0014-MP.tex-
--
PNLD0014-MP.tex:\Image{Território indígena Yanomami (Zeljko; Domínio Público)}{PNLD0014-14.png}
PNLD0014-MP.tex-
PNLD0014-MP.tex:\Image{Localização do território Yanomami (Javierfv1212; CC-BY-SA 3.0)}{PNLD0014-15.png}
PNLD0014-MP.tex-
--
PNLD0014-MP.tex:\SideImage{Folha de rosto da primeira edição do romance \emph{Úrsula} publicado em 1859. (San' Luiz; Domínio Público)}{PNLD0014-13.png}
PNLD0014-MP.tex-
--
PNLD0015-MP.tex:\Image{Selos comemorativos de 1987 com imagem de Gabriel Soares de Sousa (Wikipedia Commons; Domínio Público )}{PNLD0015-03.png}
PNLD0015-MP.tex-
--
PNLD0015-MP.tex:\SideImage{Propaganda da década de 1970 incentivando o adentramento do 
PNLD0015-MP.tex-capital no território amazônico. A que se refere as ``lendas'' 
--
PNLD0015-MP.tex:\Image{Vista de Olinda, Frans Post, 1647 (Acervo digital da Biblioteca Nacional; Domínio Público )}{PNLD0015-04.png}
PNLD0015-MP.tex-
--
PNLD0015-MP.tex:\SideImage{Gravura reproduzindo o monstro marinho de Gandavo e a forma 
PNLD0015-MP.tex-como foi capturado.}{PNLD0015-10}
--
PNLD0015-MP.tex:\Image{Primeira edição, organizada e revisada pelo historiador brasileiro Adolfo de Varnhagen (1816--1878), 1851 (Biblioteca Brasiliana Guita e José Mindlin; Domínio Público)}{PNLD0015-05.png}
PNLD0015-MP.tex-
--
PNLD0015-MP.tex:\Image{Viagens marítimas de Hans Staden feitas em vários momentos em 1547 de Portugal/Espanha para o Rio de Janeiro e outros lugares do Brasil. (Acervo digital da Biblioteca Nacional; Domínio Público )}{PNLD0015-08.png}
PNLD0015-MP.tex-
--
PNLD0015-MP.tex:\SideImage{A edição de 1851 e a de 1879 de <<Tratado Descritivo do Brazil em 1581>> são finalizadas com 270 comentários e observações de Francisco Adolpho de Varnhagen (Acervo digital da Biblioteca Nacional; Domínio Público )}{PNLD0015-06.png}
PNLD0015-MP.tex-
--
PNLD0016-MP.tex:\Image{Foto de Luiz Gama (Arquivo Nacional; Domínio Público)}{PNLD0016-03.png}
PNLD0016-MP.tex-
--
PNLD0016-MP.tex:\SideImage{Desenho de Luiz Gama, publicado no jornal "O Mequetrefe" do Rio de Janeiro, em 1882, feito pelo escritor e cartunista Raul Pompeia. (Raul Pompeia; Domínio Público)}{PNLD0016-05.png}
PNLD0016-MP.tex-
--
PNLD0016-MP.tex:\SideImage{Documento original da carta que Luiz Gama escreveu em julho de 1880 a Lúcio de Mendonça -- Parte 1. (Biblioteca Nacional; Domínio Público)}{PNLD0016-09.png}
PNLD0016-MP.tex-
PNLD0016-MP.tex:\SideImage{Parte 2 da carta. (Biblioteca Nacional; Domínio Público)}{PNLD0016-10.png}
PNLD0016-MP.tex-
--
PNLD0016-MP.tex:%\Image{Última parte da carta. (Biblioteca Nacional; Domínio Público)}{PNLD0016-12.png}
PNLD0016-MP.tex-
--
PNLD0016-MP.tex:%\Image{Parte 3 da carta. (Biblioteca Nacional; Domínio Público)}{PNLD0016-11.png}
PNLD0016-MP.tex-
--
PNLD0016-MP.tex:\Image{Princesa Isabel, Conde D'Eu e Machado de Assis na missa em celebração da Abolição da Escravatura. (Antonio Luiz Ferreira; Domínio Público)}{PNLD0016-08.png}
PNLD0016-MP.tex-
--
PNLD0016-MP.tex:\Image{Quadro "Abolição da Escravatura", de Victor Meirelles (1888) (Brasiliana Iconográfica; Domínio Público)}{PNLD0016-06.png}
PNLD0016-MP.tex-
--
PNLD0016-MP.tex:\Image{Missa campal celebrada em ação de graças pela Abolição da Escravatura no Brasil, 1888. Campo de São Cristóvão, Rio de Janeiro-RJ, Brasil. (Brasiliana Iconográfica; Domínio Público)}{PNLD0016-07.png}
PNLD0016-MP.tex-
--
PNLD0016-MP.tex:\SideImage{Busto em homenagem a Luiz Gama,  situado no Largo do Arouche, na cidade de São Paulo. Inaugurado em 1884. (Everton Zanella Alvarenga; CC0)}{PNLD0016-04.png}
PNLD0016-MP.tex-
--
PNLD0018-MP.tex:\Image{Retrato do autor, pintado em 1831 por Moritz Oppenheim (Moritz Oppenheim; Domínio público)}{PNLD0018-03.png}
PNLD0018-MP.tex-
--
PNLD0018-MP.tex:\Image{Retrato de Heine, desenhado em 1829 (Bibliothek des allgemeinen und praktischen Wissens. Bd. 5; Domínio público)}{PNLD0018-05.png}
PNLD0018-MP.tex-
--
PNLD0018-MP.tex:\Image{Monumento em homenagem ao autor, localizado em Brocken, na Alemanha (Stefan Schäfer, Lich ; CC-BY-SA-4.0)}{PNLD0018-04.png}
PNLD0018-MP.tex-
--
PNLD0018-MP.tex:\Image{Ilustração para o livro, feito por Max Liebermann para edição de 1922. (Max Liebermann; Domínio público)}{PNLD0018-06.png}
PNLD0018-MP.tex-
--
PNLD0018-MP.tex:\Image{Quadro \emph{Abraão e Sarah visitados por três anjos} (National Trust; Domínio público)}{PNLD0018-10.png}
PNLD0018-MP.tex-
--
PNLD0018-MP.tex:\Image{Ilustração para o livro, feita por Josef Budko para edição de 1921. (Josef Budko; Domínio público)}{PNLD0018-07.png}
PNLD0018-MP.tex-
--
PNLD0018-MP.tex:\Image{Ilustração de Josef Budko. (Josef Budko; Domínio público)}{PNLD0018-08.png}
PNLD0018-MP.tex-
--
PNLD0018-MP.tex:\Image{Ilustração de Josef Budko. (Josef Budko; Domínio público)}{PNLD0018-09.png}
PNLD0018-MP.tex-
--
PNLD0018-MP.tex:\Image{Aumento do número de refugiados por ano. (Fonte: UNHCR; Divulgação)}{PNLD0018-20}
PNLD0018-MP.tex-
--
PNLD0018-MP.tex:\Image{Captura de tela do site liveuamap.com}{PNLD0018-21.png}
PNLD0018-MP.tex-
--
PNLD0018-MP.tex:\SideImage{Cidade de Guernica destruída após bombardeio, 1937. Fonte: Wikipedia. CC-BY-SA 3.0}{PNLD0018-22.jpg}
PNLD0018-MP.tex-
--
PNLD0020-MP.tex:\Image{Harriet Ann Jacobs em 1894 (Jean Fagan Yellin; Domínio Público)}{PNLD0020-03.png}
PNLD0020-MP.tex-
--
PNLD0020-MP.tex:\SideImage{Vídeo sobre o fluxo dos navios negreiros do século \textsc{xv} até o fim da 
PNLD0020-MP.tex-escravidão nos \textsc{eua}. Captura de tela do vídeo. 
--
PNLD0020-MP.tex:\Image{``Dança no campo'' de Pierre"-Auguste Renoir, 1833. (Pintura localizada no Museu de Orsay, Paris; Domínio Público)}{PNLD0020-12}
PNLD0020-MP.tex:\Image{``O beijo'' de Gustav Klimt, 1908. (Pintura localizada no museu do Palácio Belvedere, Viena; Domínio Público)}{PNLD0020-11}
PNLD0020-MP.tex:\Image{``Pequeno senhor que eu amo'', de Julien de Villeneuve, 1840. (Pintura localizada no Museu de Aquitaine, Bordéus; Divulgação)}{PNLD0020-13}
PNLD0020-MP.tex:\Image{``Namorados'', de Ismael Nery, 1927. (Coleção particular, Rio de Janeiro; Domínio Público)}{PNLD0020-14}
PNLD0020-MP.tex-
--
PNLD0020-MP.tex:\Image{Publicação original do livro, em 1861. (Documenting the American South; Domínio Público)}{PNLD0020-09.png}
PNLD0020-MP.tex-
--
PNLD0020-MP.tex:\Image{Retrato do senhor de escravos James Norcom (State Archives of North Carolina Raleigh, NC; CC0)}{PNLD0020-07.png}
PNLD0020-MP.tex-
--
PNLD0020-MP.tex:\Image{Anúncio feito pelo senhor de Harriet, Norcom, com uma recompensa de 100 dólares para quem a capturasse. No cartaz, há uma descrição de sua aparência física e a explicação de ela havia fugido. Há também o aviso de que poderão sofrer penalidades os que a ajudarem em sua fuga. (State Archives of North Carolina Raleigh, NC; CC0)}{PNLD0020-05.png}
PNLD0020-MP.tex-
--
PNLD0020-MP.tex:\Image{Casa onde morou Harriet Ann Jacobs, em Cambridge, Massachusetts. (Midnightdreary; CC-BY-SA-3.0)}{PNLD0020-04.png}
PNLD0020-MP.tex-
--
PNLD0020-MP.tex:\Image{Foto de 1864 da escola fundada por Harriet Jacobs, que oferecia ensino gratuito a crianças negras. (Robert Langmuir African American Photograph Collection; Domínio Público)}{PNLD0020-08.png}
PNLD0020-MP.tex-
--
PNLD0020-MP.tex:\Image{Retrato de Harriet, desenhado com lápis grafite por Keith White em 1994. (State Archives of North Carolina Raleigh, NC; CC0)}{PNLD0020-06.png}
PNLD0020-MP.tex-
--
PNLD0020-MP.tex:\Image{"Eu era propriedade da filha deles" -- Harriet Jacobs. Essa foto, de uma menina escravizada cuidando da filha do senhor de escravos, mostra como crianças eram obrigadas a cuidar de outras crianças. (State Archives of Florida, Florida Memory; Domínio Público)}{PNLD0020-10.png}
PNLD0020-MP.tex-
--
PNLD0024-MP.tex:\Image{Monteiro Lobato em meados de 1920 (Publicado na coleção ``Nosso Século'' (1980) da Editora Abril -- volume relativo a 1910-1930, página 186; Domínio Público)}{PNLD0024-03.png}
PNLD0024-MP.tex-
--
PNLD0024-MP.tex:\SideImage{O livro ``Cidades Mortas'', originalmente publicado em 1919, reúne os primeiros escritos do autor (Internet Archive; Creative Commons CC-BY)}{PNLD0024-05.png}
PNLD0024-MP.tex-
--
PNLD0024-MP.tex:\SideImage{Por meio dessa personagem, passava-se às crianças noções de higiene e saneamento (Biblioteca Brasiliana Guita e José Mindlin; Domínio Público)}{PNLD0024-11.png}
PNLD0024-MP.tex-
--
PNLD0024-MP.tex:\Image{Por meio dessa personagem, passava-se às crianças noções de higiene e saneamento. (Biblioteca Brasiliana Guita e José Mindlin; Domínio Público )}{PNLD0024-10.png}
PNLD0024-MP.tex-
--
PNLD0024-MP.tex:\Image{Por meio dessa personagem, passava-se às crianças noções de higiene e saneamento (Biblioteca Brasiliana Guita e José Mindlin; Domínio Público )}{PNLD0024-09.png}
PNLD0024-MP.tex-
--
PNLD0024-MP.tex:\SideImage{Selos comemorativos de 1973 com as personagens do Sítio do Pica-Pau Amarelo: Emília, Tia Nastácia, Narizinho, Pedrinho, Quindim, Visconde de Sabugosa, Rabicó, Burro Falante, Dona Benta (Correios do Brasil; Domínio Público )}{PNLD0024-04.png}
PNLD0024-MP.tex-
--
PNLD0024-MP.tex:\SideImage{O autor traduziu e adaptou diversas obras, dentre as quais ``Por quem os sinos dobram'' de Ernest Hemingway (Hemeroteca Digital – Biblioteca Nacional; Domínio Público )}{PNLD0024-08.png}
PNLD0024-MP.tex-
--
PNLD0024-MP.tex:\SideImage{Considerada obra-prima do autor, publicada em 1920 (Internet Archive; Domínio Público )}{PNLD0024-06.png}
PNLD0024-MP.tex-
--
PNLD0024-MP.tex:\Image{Primeiro livro da série de histórias do autor dedicada às crianças (Biblioteca Brasiliana Guita e José Mindlin; Domínio Público)}{PNLD0024-07.png}
PNLD0024-MP.tex-
--
PNLD0026-MP.tex:\Image{Machado de Assis (de óculos), aos 67 anos, com Joaquim Nabuco, 1906 (Augusto Malta; Domínio Público)}{PNLD0026-03.png}
PNLD0026-MP.tex-
--
PNLD0026-MP.tex:\Image{Casa em que morou Machado de Assis no Cosme Velho (Acervo Digital da Biblioteca Nacional; Domínio Público)}{PNLD0026-04.png}
PNLD0026-MP.tex-
--
PNLD0026-MP.tex:\SideImage{Desenho do escritor brasileiro e primeiro editor de Machado de Assis, Francisco de Paula Brito  (Acervo Digital da Biblioteca Nacional; Domínio Público)}{PNLD0026-10.png}
PNLD0026-MP.tex-
--
PNLD0026-MP.tex:\Image{Machado de Assis aos 25 anos (Acervo Digital da Biblioteca Nacional; Domínio Público)}{PNLD0026-05.png}
PNLD0026-MP.tex-
--
PNLD0026-MP.tex:\Image{Manuscrito do capítulo ``A moeda de Vespasiano'' do livro ``Memórias Póstumas de Brás Cubas'' de Machado de Assis, 1884 (Acervo Digital da Biblioteca Nacional; Domínio Público)}{PNLD0026-07.png}
PNLD0026-MP.tex-
--
PNLD0026-MP.tex:\Image{Estátua de bronze de Machado de Assis na fachada do prédio da Academia Brasileira de Letras (Escultura de Humberto Cozzo; Domínio Público)}{PNLD0026-08.png}
PNLD0026-MP.tex-
--
PNLD0026-MP.tex:\Image{Panorama em 3 partes da Enseada do Botafogo, entre 1860 e 1879 (Acervo Digital da Biblioteca Nacional; Domínio Público)}{PNLD0026-09.png}
PNLD0026-MP.tex-
--
PNLD0026-MP.tex:\Image{Machado de Assis acompanhado de colegas como Pereira Passos, Joaquim Nabuco, José Américo dos Santos, Aloysio de Carvalho, Lafayette Rodrigues Pereira e Gastão da Cunha. (Augusto Malta; Domínio Público)}{PNLD0026-06.png}
PNLD0026-MP.tex-
--
PNLD0027-MP.tex:	\Image{Abertura do documentário \emph{Crônicas do crack}, de Luis Marra, disponível 
PNLD0027-MP.tex-	no www.youtube.com/watch?v=47NIEsOO6qA}{PNLD0027-11}
--
PNLD0027-MP.tex:\Image{Luis Marra, autor de \textit{Crônicas do Crack}.}{PNLD0027-15.png}
PNLD0027-MP.tex-
--
PNLD0027-MP.tex:\Image{O livro aborda a temática da dependência química a partir 
PNLD0027-MP.tex-da perspectiva do indivíduo (Marco Gomes; CC-BY 2.0)}{PNLD0027-04.png}
--
PNLD0027-MP.tex:\SideImage{O crack inibe a recaptação da dopamina alterando o 
PNLD0027-MP.tex-funcionamento dos neurônios (Autor Desconhecido; Domínio Público)}{PNLD0027-05.png}
--
PNLD0027-MP.tex:\Image{Substâncias químicas e o corpo humano (Drug Disposal; Domínio Público)}{PNLD0027-06.png}
PNLD0027-MP.tex-
--
PNLD0027-MP.tex:\Image{Dependente químico mostrando cachimbo  (Marco Gomes; CC-BY 2.0)}{PNLD0027-08.png}
PNLD0027-MP.tex-
--
PNLD0027-MP.tex:\Image{Moradores de rua no centro de São Paulo (Marco Gomes; CC-BY 2.0)}{PNLD0027-07.png}
PNLD0027-MP.tex-
--
PNLD0027-MP.tex:\Image{Dependentes químicos na região da Cracolândia (Luciana Marinho/Flickr; CC-BY 2.0)}{PNLD0027-09.png}
PNLD0027-MP.tex-
--
PNLD0027-MP.tex:\Image{Entorno da Cracolândia (Marco Gomes; CC-BY 2.0)}{PNLD0027-10.png}
PNLD0027-MP.tex-
--
PNLD0029-MP.tex:\Image{Anne Ballester Soares (Autor Desconhecido; Acervo da Autora)}{PNLD0029-03.png}
PNLD0029-MP.tex-
--
PNLD0029-MP.tex:\Image{Mapa do território Yanomami no Brasil e Venezuela, 2012. (Wikimedia Commons; CC-BY-SA 3.0)}{PNLD0029-05.png}
PNLD0029-MP.tex-
--
PNLD0029-MP.tex:\Image{Xapono, casa comunal habitada pelo Yanomami. (Wikimedia Commons; Domínio Público)}{PNLD0029-04.png}
PNLD0029-MP.tex-
--
PNLD0029-MP.tex:\Image{Mulheres Yanomami do lado amazônico. (Wikimedia Commons; CC-BY-SA 3.0)}{PNLD0029-06.png}
PNLD0029-MP.tex-
--
PNLD0029-MP.tex:\Image{Mulher Yanomami tecendo cesta, 1999. (Cmacauley; CC-BY-SA 3.0)}{PNLD0029-07.png}
PNLD0029-MP.tex-
--
PNLD0030-MP.tex:\Image{A autora, Orides Fontela. (Arquivo da autora.)}{PNLD0030-03.png}
PNLD0030-MP.tex-
--
PNLD0030-MP.tex:\SideImage{Heloísa Buarque de Hollanda é autora de "Pensamento feminista brasileiro" e organizadora do livro "As 29 poetas hoje", lançado em 2021. (Produção Cultural no Brasil; CC-BY-SA 2.0)}{PNLD0030-08.png}
PNLD0030-MP.tex-
--
PNLD0030-MP.tex:\SideImage{A indiana Rupi Kaur é uma das responsáveis por trazer visibilidade para mulheres poetas através das redes sociais. (Baljit Singh; CC-BY-SA 4.0)}{PNLD0030-09.png}
PNLD0030-MP.tex-
--
PNLD0030-MP.tex:  \SideImage{Alguns diretores do Cinema Novo: movimento cinematográfico brasileiro marcante nos anos 1960 e 1970 e que se destacou por sua crítica a desigualdade social e a Ditadura Civil-Militar. (Johnsmith22who; CC-BY-SA 4.0)}{PNLD0030-12.png}
PNLD0030-MP.tex-
--
PNLD0030-MP.tex:\Image{O fim da década de 1970 e começo de 1980 no Brasil foi marcado pelo movimento das Diretas Já, que reuniu milhares de pessoas nas ruas se manifestando a favor da volta das eleições. (Arquivo da Agência Brasil; CC-0)}{PNLD0030-10.png}
PNLD0030-MP.tex-
--
PNLD0030-MP.tex:% \SideImage{O Teatro de Arena foi um dos mais importantes grupos teatrais brasileiros das décadas de 1950 e 1960. (Henrique Artuni; CC-BY-SA 2.0)}{PNLD0030-13.png}
PNLD0030-MP.tex-
PNLD0030-MP.tex:\SideImage{A obra "Seja marginal, seja herói" de Hélio Oiticica é um marco do movimento de cultura marginal, que passou a fazer parte do debate cultural brasileiro a partir do final de 1968. (Wally Gobetz; CC-BY-NC-ND 2.0)}{PNLD0030-11.png}
PNLD0030-MP.tex-
--
PNLD0030-MP.tex:  \Image{Caetano Veloso no III Festival da Música Popular (Arquivo Nacional; Domínio Público)}{PNLD0030-15.png}
PNLD0030-MP.tex-
--
PNLD0030-MP.tex:  \SideImage{Gilberto Gil e Nana Caymmi no III Festival da Música Popular (Arquivo Nacional; Domínio Público)}{PNLD0030-14.png}
PNLD0030-MP.tex-
--
PNLD0030-MP.tex:\Image{Orides Fontela (1940-1998) (Inez Guerreiro; Inez Guerreiro)}{PNLD0030-04.png}
PNLD0030-MP.tex-
--
PNLD0030-MP.tex:\Image{Parte da frente de bilhete escrito à mão por Orides Fontela. (Arquivo da autora.)}{PNLD0030-06.png}
PNLD0030-MP.tex-
--
PNLD0030-MP.tex:\Image{Verso de bilhete escrito à mão por Orides Fontela. (Arquivo da autora.)}{PNLD0030-07.png}
PNLD0030-MP.tex-
--
PNLD0030-MP.tex:\SideImage{Orides Fontela (1940-1998) (Arquivo da autora.)}{PNLD0030-05.png}
PNLD0030-MP.tex-
--
PNLD0033-MP.tex:\SideImage{Escultura de Patativa do Assaré, Fortaleza, Ceará. (Autoria Desconhecida; CC-BY-SA 2.0)}{PNLD0033-03.png}
PNLD0033-MP.tex-
--
PNLD0033-MP.tex:\Image{Literatura de Cordel. (Diego Dacal/Wikimedia Commons; CC-BY-SA 2.0)}{PNLD0033-06.png}
PNLD0033-MP.tex-
--
PNLD0033-MP.tex:\Image{Flor de mandacaru registrada em trilha na Bahia. (Wellscte/Wikimedia Commons; CC-BY-SA 4.0)}{PNLD0033-08.png}
PNLD0033-MP.tex-
--
PNLD0033-MP.tex:\Image{Xilogravura da Artista plástica Yolanda Carvalho (Rafael Nolêto/Wikimedia Commons; CC-BY-SA 3.0)}{PNLD0033-10.png}
PNLD0033-MP.tex-
--
PNLD0033-MP.tex:\Image{Município de Assaré em vermelho, 2006. (Raphael Lorenzeto de Abreu/Wikimedia Commons; CC-BY-SA 3.0)}{PNLD0033-05.png}
PNLD0033-MP.tex-
--
PNLD0033-MP.tex:\Image{Memorial Patativa do Assaré, em Assaré, Ceará. (Allan Patrick/Flickr; CC-BY-SA 2.0)}{PNLD0033-04.png}
PNLD0033-MP.tex-
--
PNLD0033-MP.tex:\Image{Ilustração da capa do cordel ``O ABC da Cachaça'' (Thomas Fisher Rare Book Library/Flickr; CC-BY-NC-SA 2.0)}{PNLD0033-07}
PNLD0033-MP.tex-
--
PNLD0033-MP.tex:\Image{Capa de ``A Xilogravura Popular e a Literatura de Cordel'', 1985 (Thomas Fisher Rare Book Library/Flickr; CC-BY-NC-SA 2.0)}{PNLD0033-09.png}
PNLD0033-MP.tex-
--
PNLD0035-MP.tex:\Image{Escultura de Safo, Museu Bourdelle, em Paris (Jean-Pierre Dalbéra; CC-BY-SA 2.0)}{PNLD0035-03}
PNLD0035-MP.tex-
--
PNLD0035-MP.tex:\SideImage{Ânfora tirrena de produção ateniense, período arcaico, 570--560 a.C (Wikimedia Commons; CC-BY-SA 3.0)}{PNLD0035-04}
PNLD0035-MP.tex-
--
PNLD0035-MP.tex:\Image{À direita acima na ânfora, Afrodite está sentada de luto, Fthonos encostado nela. Abaixo está Oineus, cetro na mão esquerda. Abaixo da sala, Teseu e Peleu estão de luto (Digital LIMC; Domínio Público)}{PNLD0035-05}
PNLD0035-MP.tex-
--
PNLD0035-MP.tex:\Image{O desenho mostra Afrodite nua segurando a pequena figura de Eros perto dela enquanto ele tenta atirar uma flecha de seu arco (Britannica; Domínio Público)}{PNLD0035-06}
PNLD0035-MP.tex-
--
PNLD0035-MP.tex:\Image{Estátua da deusa Ártemis. Cópia romana de uma estátua original grega de Leocares (Localizada no Museu do Louvre, em Paris; CC BY-SA 4.0)}{PNLD0035-16}
PNLD0035-MP.tex-
--
PNLD0035-MP.tex:\Image{Busto de Afrodite, princípio do século \textsc{iv} a. C, Museu Arqueológico Nacional de Atenas (Jerónimo Roure Pérez; CC-BY-SA 4.0)}{PNLD0035-07}
PNLD0035-MP.tex-
--
PNLD0035-MP.tex:\SideImage{\emph{Noite estrelada sobre o Ródano} (1888), Vincent Van Gogh (Localizado no Museu de Orsay, em Paris; Domínio Público)}{PNLD0035-11}
PNLD0035-MP.tex-
--
PNLD0035-MP.tex:\SideImage{\emph{Os Amantes} (1928), René Magritte (Localizado no Museu de Arte Moderna de Nova York; Divulgação)}{PNLD0035-12}
PNLD0035-MP.tex-
PNLD0035-MP.tex:\SideImage{\emph{Monumento às Bandeiras} (1953), Victor Brecheret, Cidade de São Paulo (Domínio público)}{PNLD0035-13}
PNLD0035-MP.tex-
PNLD0035-MP.tex:\SideImage{Metrô de São Paulo em horário de pico (Divulgação)}{PNLD0035-14}
PNLD0035-MP.tex-
--
PNLD0035-MP.tex:\Image{Afrodite em esculturas gregas antigas (Projekt Runeberg; Domínio Público)}{PNLD0035-08}
PNLD0035-MP.tex-
--
PNLD0035-MP.tex:\Image{Representação da poetisa grega segurando a carta que vai enviar ao amante Phaon, antes de se matar (Museu do Louvre; Domínio Público)}{PNLD0035-09}
PNLD0035-MP.tex-
--
PNLD0035-MP.tex:\Image{\textit{Baco e Midas} (1629), Nicolas Poussin (Localizado na Pinacoteca de Munique; Domínio Público)}{PNLD0035-15}
PNLD0035-MP.tex-
--
PNLD0035-MP.tex:\Image{Pintura a óleo de Safo e Alceu, 1881 (Google Art Project; Domínio Público)}{PNLD0035-10}
PNLD0035-MP.tex-
--
PNLD0036-MP.tex:\Image{Retrato de Marcel Schwob em 1902. (Paul Boyer; Domínio Público)}{PNLD0036-03.png}
PNLD0036-MP.tex-
--
PNLD0036-MP.tex:\Image{\textit{A cruzada das crianças}, na ilustração de Gustave Dorée, representa a mítica cruzada que inspirou Schwob na escrita do presente livro. (Domínio público)}{PNLD0036-12}
PNLD0036-MP.tex-
--
PNLD0036-MP.tex:\SideImage{Trovadores da Idade Média, aedos da Grécia antiga, griôs
PNLD0036-MP.tex-da África e repentistas do Nordeste brasileiro têm comum a arte de contar histórias acompanhados de música. (Autor desconhecido. Domínio público.)}{PNLD0036-13}
--
PNLD0036-MP.tex:\Image{``O triunfo da morte'' (1562), de Pieter Bruegel, representa a devastação causada pelas
PNLD0036-MP.tex-guerras e pela doença na Idade Média europeia. (Museu do Prado de Madri; Domínio Público)}{PNLD0036-11}.
--
PNLD0036-MP.tex:\SideImage{Retrato do autor desenhado por Theodore Simson em 1905, ano de seu falecimento. (Theodore Frederick Spicer Simson; Domínio Público)}{PNLD0036-04.png}
PNLD0036-MP.tex-
--
PNLD0036-MP.tex:% \SideImage{Publicação de 1927 de Marcel Schwob, com seus escritos da juventude. (Gallica; Domínio Público)}{PNLD0036-10.png}
PNLD0036-MP.tex-
--
PNLD0036-MP.tex:\Image{Mapa da rota das Cruzadas, expedições militares organizadas com a benção da Igreja Católica. (Rowanwindwhistler; CC-BY-SA-3.0)}{PNLD0036-05.png}
PNLD0036-MP.tex-
--
PNLD0036-MP.tex:\Image{Manuscrito medieval de 1474 que buscou representar a Cruzada Popular de Pedro, o Eremita, de 1096, o movimento extraoficial da Primeira Cruzada (Biblioteca Nacional da França; Domínio Público)}{PNLD0036-06.png}
PNLD0036-MP.tex-
--
PNLD0036-MP.tex:\Image{Ilustração do flautista de Hamelin, por Kate Greenaway. (Kate Greenaway; Domínio Público)}{PNLD0036-07.png}
PNLD0036-MP.tex-
--
PNLD0036-MP.tex:\SideImage{Representação do Papa Inocêncio \textsc{iii} feita entre 1275 e 1230. (\emph{Grandes Chroniques de France}, Bibliothèque Sainte"-Geneviève; CC-BY-SA-3.0)}{PNLD0036-08.png}
PNLD0036-MP.tex-
--
PNLD0036-MP.tex:\Image{O quadro de Manuel de La Cruz, de 1788, representa uma visão do Papa Inocêncio \textsc{iii}, relacionada com São Francisco de Assis. (Manuel de la Cruz; Domínio Público)}{PNLD0036-09.png}
PNLD0036-MP.tex-
--
PNLD0038-MP.tex:\Image{Gravura de William Wells Brown, 1847 (Autor Desconhecido; Domínio Público )}{PNLD0038-03.png}
PNLD0038-MP.tex-
--
PNLD0038-MP.tex:\Image{Representação do autor enquanto personagem principal do livro (Internet Archive; Domínio Público)}{PNLD0038-04.png}
PNLD0038-MP.tex-
--
PNLD0038-MP.tex:\SideImage{Na legenda da ilustração, o poema de Thomas Campbell \emph{United States! your banner wears/ Two emblems--one of fame:/ Alas! the other that it bears/ Reminds us of your shame:/ Your banner's constellation types/ White freedom with its stars, /But what's the meaning of the stripes?/ They mean your negroes' scars}. (Internet Archive; Domínio Público)}{PNLD0038-07.png}
PNLD0038-MP.tex-
--
PNLD0038-MP.tex:\Image{O traficante de escravos e o autor conduzindo uma grupo de escravos para o mercado do sul (Internet Archive; Domínio Público)}{PNLD0038-05.png}
PNLD0038-MP.tex-
--
PNLD0039-MP.tex:\Image{Foto de Strindberg após seu aniversário de 50 anos. (Autor desconhecido; Domínio Público)}{PNLD0039-03.png}
PNLD0039-MP.tex-
--
PNLD0039-MP.tex:\Image{O diretor Ingmar Bergman, conterrâneo de Strindberg, no set do 
PNLD0039-MP.tex-filme \textit{Morangos Silvestres}. (Louis Huch; Domínio Público)}{PNLD0039-09.png}
--
PNLD0039-MP.tex:\Image{Retrato de Strindberg com 25 anos (Studio de Math. S. Hansen; Domínio Público)}{PNLD0039-04.png}
PNLD0039-MP.tex-
--
PNLD0039-MP.tex:\Image{Vista da ilha Öland, na Suécia, no mar Báltico, localização 
PNLD0039-MP.tex-aproximada da fictícia ilha de Hemsö. (Arnold Paul; CC-BY-SA 2.5)}{PNLD0039-06.png}
--
PNLD0039-MP.tex:\Image{Retrato de Strindberg pintado por Richard Bergh em 1905 (Richard Bergh; Domínio Público)}{PNLD0039-05.png}
PNLD0039-MP.tex-
--
PNLD0039-MP.tex:\Image{O diretor Ingmar Bergman e o ator principal do filme \textit{Morangos Silvestres}, Victor Sjöström. (Åke Blomquist; Domínio Público)}{PNLD0039-10.png}
PNLD0039-MP.tex-
--
PNLD0039-MP.tex:\Image{Ilha de Utö, localizada no arquipélago de Estocolmo. Ao fundo, 
PNLD0039-MP.tex-vemos o mar Báltico. (Arild Vågen ; CC-BY-SA 2.0)}{PNLD0039-07.png}
--
PNLD0039-MP.tex:\Image{Mapa do arquipélago de Estocolmo (Demis map server; Domínio Público)}{PNLD0039-08.png}
PNLD0039-MP.tex-
--
PNLD0040-MP.tex:\SideImage{Foto de Strindberg após seu aniversário de 50 anos. (Autor desconhecido; Domínio Público)}{PNLD0040-03.png}
PNLD0040-MP.tex-
--
PNLD0040-MP.tex:\SideImage{Quadro "Thor luta com os gigantes", pintado por Mårten Eskil Winge em 1872. (Mårten Eskil Winge; Domínio Público)}{PNLD0040-06.png}
PNLD0040-MP.tex-
--
PNLD0040-MP.tex:\SideImage{Representação de Freir, divindade nórdica da paz e da fertilidade, das colheitas e dos casamentos (Jacques Reich; Domínio Público)}{PNLD0040-10.png}
PNLD0040-MP.tex-
--
PNLD0040-MP.tex:\Image{Ilustração de 1850 representa Odim, uma das mais importantes divindades nórdicas. (Otto Henrik Wallgren; Domínio Público)}{PNLD0040-08.png}
PNLD0040-MP.tex-
--
PNLD0040-MP.tex:\Image{Desenho a lápis representa Odim, o deus da guerra segundo a mitologia escandinava. (Victor Villalobos; CC-BY-SA 4.0)}{PNLD0040-09.png}
PNLD0040-MP.tex-
--
PNLD0040-MP.tex:\Image{Retrato de Strindberg com 25 anos. (Studio de Math. S. Hansen; Domínio Público)}{PNLD0040-04.png}
PNLD0040-MP.tex-
--
PNLD0040-MP.tex:\Image{Retrato de Strindberg pintado por Richard Bergh em 1905 (Richard Bergh; Domínio Público)}{PNLD0040-05.png}
PNLD0040-MP.tex-
--
PNLD0040-MP.tex:\Image{Escultura Viking localizada em Largs, Reino Unido. (Dave Hitchborne; CC-BY-SA 2.0)}{PNLD0040-07.png}
PNLD0040-MP.tex-
--
PNLD0041-MP.tex:\Image{Retrato de Johann Ludwig Tieck, 1828. (Vogel, C.; Domínio Público)}{PNLD0041-03.png}
PNLD0041-MP.tex-
--
PNLD0041-MP.tex:\SideImage{Casa Ludwig Tieck de 1819 a 1842, Saxônia, Alemanha (Lupus in Saxonia; CC-BY-SA 4.0)}{PNLD0041-05.png}
PNLD0041-MP.tex-
--
PNLD0041-MP.tex:\Image{O gazebo, 1861 (MDZ München; Domínio Público)}{PNLD0041-06.png}
PNLD0041-MP.tex-
--
PNLD0041-MP.tex:\SideImage{Ilustração para a peça lírica ``Vida e morte de Santa Genoveva'' (Gbi.bytos/Wikimedia Commons; CC-BY-SA 4.0)}{PNLD0041-07.png}
PNLD0041-MP.tex-
--
PNLD0041-MP.tex:\SideImage{``Gato de Botas. Contos de fadas infantis em três atos'', Berlim, 1797, por Ludwig Tieck. (Klaus Günzel: Die deutschen Romantiker. Artemis, Zürich, 1995.; Domínio Público)}{PNLD0041-08.png}
PNLD0041-MP.tex-
--
PNLD0041-MP.tex:\SideImage{Ilustração para ``A vida do famoso imperador Abraham Tonelli'', 1920 (Rolf von Hoerschelmann; Domínio Público)}{PNLD0041-09.png}
PNLD0041-MP.tex-
--
PNLD0041-MP.tex:\Image{Ilustração d'une Berceuse de Tieck (canção de ninar) por Ludwig Richter (Gbi.bytos/Wikimedia Commons; CC-BY-SA 4.0)}{PNLD0041-10.png}
PNLD0041-MP.tex-
--
PNLD0041-MP.tex:\Image{Wilhelm Grimm e Jacob Grimm em 1847, Daguerreótipo. (Hermann Blow; Domínio Público)}{PNLD0041-04.png}
PNLD0041-MP.tex-
--
PNLD0042-MP.tex:\Image{Tolstói em maio de 1908, quatro meses antes de seu aniversário de 80 anos (fotografado em Iasnaia Poliana por Sergei Mikhailovitch Prokudin"-Gorski; a primeira foto colorida tirada oficialmente na Rússia) (Serguéi Mijáilovich Prokudin"-Gorskii; Domínio Público)}{PNLD0042-03.png}
PNLD0042-MP.tex-
--
PNLD0042-MP.tex:\Image{Folha de rosto da edição de 1895, no original russo. (Anakay; Domínio Público)}{PNLD0042-06.png}
PNLD0042-MP.tex-
--
PNLD0042-MP.tex:\SideImage{Retrato de Pedro \textsc{i} (1672--1725) (Arkhangelskoye Palace; Domínio Público)}{PNLD0042-07.png}
PNLD0042-MP.tex-
--
PNLD0042-MP.tex:\Image{Família imperial russa em 1913 (Boasson and Eggler St. Petersburg Nevsky 24.; Domínio Público)}{PNLD0042-08.png}
PNLD0042-MP.tex-
--
PNLD0042-MP.tex:\Image{Retrato do século \textsc{xviii} de um mujique, camponês russo. (Dmitrismirnov; Domínio Público)}{PNLD0042-09.png}
PNLD0042-MP.tex-
--
PNLD0042-MP.tex:\SideImage{Pintura de Tolstói vestido com roupas camponesas, por Ilya Repin (1901) (Ilya Repin; Domínio Público)}{PNLD0042-05.png}
PNLD0042-MP.tex-
--
PNLD0042-MP.tex:\Image{O jovem Tolstói, em 1854 (Pavel Biryukov; Domínio Público)}{PNLD0042-04.png}
PNLD0042-MP.tex-
--
PNLD0042-MP.tex:\Image{Grupo de camponesas russas em 1917 (Francis Brewster Reeves; CC0)}{PNLD0042-10.png}
PNLD0042-MP.tex-
--
PNLD0043-MP.tex:\Image{Retrato de Mark Twain, 1907 (A.F. Bradley; Domínio Público)}{PNLD0043-03.png}
PNLD0043-MP.tex-
--
PNLD0043-MP.tex:\Image{Biblioteca do primeiro andar da casa de Mark Twain (Jack E. Boucher; Domínio Público)}{PNLD0043-05.png}
PNLD0043-MP.tex-
--
PNLD0043-MP.tex:\SideImage{Casa onde Mark Twain passou sua infância, Hannibal, Missouri, EUA. (Andrew Balet/Wikimedia Commons; CC-BY 2.5)}{PNLD0043-06.png}
PNLD0043-MP.tex-
PNLD0043-MP.tex:\SideImage{Casa de Mark Twain em Hartford, Connecticut, EUA. (Srett/Flickr; CC-BY-SA 2.0)}{PNLD0043-04.png}
PNLD0043-MP.tex-
--
PNLD0043-MP.tex:\SideImage{Capa da primeira edição de ``O Paraíso Perdido'' de John Milton, 1667. (Houghton Library; Domínio Público)}{PNLD0043-07.png}
PNLD0043-MP.tex-
--
PNLD0043-MP.tex:\SideImage{Capa de ``Eve's Diary'', 1906 (Wikimedia Commons; Domínio Público)}{PNLD0043-08.png}
PNLD0043-MP.tex-
--
PNLD0043-MP.tex:\SideImage{Capa de ``Extracts From Adam's Diary'', 1904 (Wikimedia Commons; Domínio Público)}{PNLD0043-09.png}
PNLD0043-MP.tex-
--
PNLD0043-MP.tex:\Image{Ilustração do livro ``Eve's Diary'', 1907 (Project Gutenberg; Domínio Público)}{PNLD0043-10.png}
PNLD0043-MP.tex-
--
PNLD0045-MP.tex:%\Image{Escultura Patativa do Assaré, Fortaleza, Ceará (Autoria Desconhecida; CC-BY-SA 2.0)}{PNLD0045-08.png}
PNLD0045-MP.tex:\Image{Retrato de João Martins de Athayde (Memórias do Cordel; CC-BY-SA
PNLD0045-MP.tex-2.0)}{PNLD0045-03.png}
--
PNLD0045-MP.tex:\Image{Literatura de Cordel (Diego Dacal/Wikimedia Commons; CC-BY-SA
PNLD0045-MP.tex-2.0)}{PNLD0045-05.png}
--
PNLD0045-MP.tex:\Image{Capa do cordel ``História de Roberto do Diabo'' (Fundação
PNLD0045-MP.tex-Casa de Rui Barbosa; Domínio Público)}{PNLD0045-06.png}
--
PNLD0045-MP.tex:\Image{Xilogravura da Artista plástica Yolanda Carvalho (Rafael
PNLD0045-MP.tex-Nolêto/Wikimedia Commons; CC-BY-SA 3.0)}{PNLD0045-10.png}
--
PNLD0045-MP.tex:\Image{Capa do cordel ``Os sofrimentos de Alzira'' (Fundação Casa de Rui
PNLD0045-MP.tex-Barbosa; Domínio Público)}{PNLD0045-07.png}
--
PNLD0045-MP.tex:\Image{Mapa de Ingá na Paraíba, cidade onde nasceu o autor (Marcos Elias de
PNLD0045-MP.tex-Oliveira Júnior/Wikimedia Commons; CC-BY-SA 3.0)}{PNLD0045-04.png}
--
PNLD0045-MP.tex:\Image{Capa de ``A Xilogravura Popular e a Literatura de Cordel'', 1985 (Thomas
PNLD0045-MP.tex-Fisher Rare Book Library/Flickr; CC-BY-NC-SA 2.0)}{PNLD0045-09.png}
--
PNLD0046-MP.tex:\SideImage{Folha de rosto da obra, original da edição de Dufart, Paris, 1796. (Gallica; Domínio Público)}{PNLD0046-05.png}
PNLD0046-MP.tex-
--
PNLD0046-MP.tex:\Image{A temática do mergulho em seu próprio quarto também foi explorada por outros artistas, como Van Gogh em seu famoso quadro ``Quarto em Arles'', de 1888. (Museu do Van Gogh; CC-BY-SA 2.0)}{PNLD0046-07.png}
PNLD0046-MP.tex-
--
PNLD0046-MP.tex:\Image{Pintura de Nikolay Karazin baseada no livro ``Crime e Castigo''  de Dostoiévski. O homem como objeto de análise é um tema comum da narrativa dessa obra e de ``Viagem em volta do meu quarto''.  (Nikolay Karazin; Domínio Público)}{PNLD0046-08.png}
PNLD0046-MP.tex-
--
PNLD0046-MP.tex:\Image{Folha de rosto da primeira edição do livro ``Anna Karenina'', de Tolstói, no original russo. O homem como objeto de análise e sua vivência em uma sociedade de aparências é um dos temas presentes.  (Manhattan Rare Books; Domínio Público)}{PNLD0046-09.png}
PNLD0046-MP.tex-
--
PNLD0046-MP.tex:\Image{Folha de rosto de ``Memórias Póstumas de Brás Cubas'' dedicada por Machado de Assis para a Biblioteca Nacional. Logo no início do livro, Machado faz referência à Xavier de Maistre, em quem se inspirou na maneira de dialogar com o leitor. (Biblioteca Nacional; Domínio Público)}{PNLD0046-10.png}
PNLD0046-MP.tex-
--
PNLD0046-MP.tex:\Image{O estilo da narrativa que se desenvolve como um bate"-papo também é encontrada em \textit{Grande Sertão: Veredas} de Guimarães Rosa. Nessa foto, é possível ver o autor em uma das viagens que fez ao sertão em 1952. (Eugênio Silva; CC0)}{PNLD0046-11.png}
PNLD0046-MP.tex-
--
PNLD0046-MP.tex:\Image{Autorretrato do autor, de aproximadamente 1825 (Xavier de Maistre; Domínio Público)}{PNLD0046-04.png}
PNLD0046-MP.tex-
--
PNLD0046-MP.tex:\Image{Estátua em homenagem as irmãos Joseph e Xavier de Maistre, localizada no castelo de Chambéry, na França. (Florian Pépellin; CC-BY-SA 4.0)}{PNLD0046-06.png}
PNLD0046-MP.tex-
--
PNLD0048-MP.tex:\Image{Retrato de Machado de Assis, 1890 (Marc Ferrez; Domínio público)}{PNLD0048-03.png}
PNLD0048-MP.tex-
--
PNLD0048-MP.tex:\SideImage{Gravura de Francisco de Paula Brito  (Autor Desconhecido, Século XIX. Pinto, Ana Flávia Magalhães. De Pele Escura e Tinta Preta: a imprensa negra do século XIX (1833-1899). Mestrado. Universidade de Brasília, 2006; Domínio público)}{PNLD0048-05.png}
PNLD0048-MP.tex-
--
PNLD0048-MP.tex:\SideImage{Morro do Livramento, Rio de Janeiro, onde Machado de Assis nasceu (a seta aponta no alto do canto direito a possível casa de nascimento), sem data. (ABL; Domínio público)}{PNLD0048-04.png}
PNLD0048-MP.tex-
--
PNLD0048-MP.tex:\SideImage{Pintura ``O vampiro'' (1897) de Philip Burne-Jones (Philip Burne-Jones; Domínio público)}{PNLD0048-07.png}
PNLD0048-MP.tex-
--
PNLD0048-MP.tex:\SideImage{Placa comemorativa para o clássico do cinema mudo Nosferatu (1922) em Wismar, na Alemanha. Partes do filme foram filmadas em Wismar. (J.H. Janßen; Domínio público)}{PNLD0048-08.png}
PNLD0048-MP.tex-
PNLD0048-MP.tex:\SideImage{O vampiro Conde Drácula movendo-se como um lagarto ao longo da parede de seu castelo. Ilustração da capa de Holloway para a décima terceira edição do romance de Bram Stoker (Londres, William Rider and Son, 1919). (Holloway; Domínio público)}{PNLD0048-06.png}
PNLD0048-MP.tex-
--
PNLD0048-MP.tex:\SideImage{Crepúsculo é um filme norte-americano de 2008, dirigido por Catherine Hardwicke, sendo o primeiro da saga, adaptado do primeiro livro de Stephenie Meyer (Flickr ; Creative Commons CC-BY-SA 2.0)}{PNLD0048-09.png}
PNLD0048-MP.tex-
--
PNLD0048-MP.tex:\SideImage{``Poética'' de Aristóteles é um conjunto de anotações de suas aulas sobre o tema da poesia e da arte em sua época. (Biblioteca da Universidade de Sidney ; Domínio público)}{PNLD0048-10.png}
PNLD0048-MP.tex-
--
PNLD0048-MP.tex:\Image{Quadro ``A liberdade guiando o povo'' de Eugène Delacroix, de 1830. (Eugène Delacroix; Domínio público)}{PNLD0048-11.png}
PNLD0048-MP.tex-
--
PNLD0048-MP.tex:Image{``Hamlet'' de William Shakespeare edição de 1866 (Internet Archive; Domínio público)}{PNLD0048-12.png}
PNLD0048-MP.tex-
--
PNLD0049-MP.tex:\SideImage{Foto de Maksim Górki (1868-1936), escritor russo (Wikipédia; Domínio público).}{PNLD0049-08}
PNLD0049-MP.tex-
--
PNLD0049-MP.tex:\SideImage{Samovar: utensílio tradicional russo usado para ferver a água do chá. Ilustração para livro de Fabio Flaks (Kalinka; Direitos cedidos à Kalinka).}{PNLD0049-04}
PNLD0049-MP.tex-
--
PNLD0049-MP.tex:\SideImage{Gato Vassíli de Nikita. Ilustração para livro de Fabio Flaks (Kalinka; Direitos cedidos à Kalinka).}{PNLD0049-09}
PNLD0049-MP.tex-
--
PNLD0049-MP.tex:\SideImage{Jeltúkhin, estorninho de Nikita. Ilustração para livro de Fabio Flaks (Kalinka; Direitos cedidos à Kalinka).}{PNLD0049-05}
PNLD0049-MP.tex-
--
PNLD0049-MP.tex:\SideImage{Retrato de Nikita imaginado por Fabio Flaks (Kalinka; Direitos cedidos à Kalinka).}{PNLD0049-10}
PNLD0049-MP.tex-
--
PNLD0049-MP.tex:\SideImage{Retrato de Aleksándr Púchkin (1799-1837), expoente da poesia russa, feito por O. Kipriénski (1872-1936) (Wikipédia russa; Domínio público).}{PNLD0049-11}
PNLD0049-MP.tex-
--
PNLD0049-MP.tex:\Image{``Camponeses à mesa.'' Foto (de 1907 a 1915): Serguei Prokúdin-Górski (1863-1944) (Wikipédia; Domínio público).}{PNLD0049-12}
PNLD0049-MP.tex-
--
PNLD0049-MP.tex:\SideImage{Alexandre II (1855-1881), imperador russo que decretou o fim do regime de servidão (1861). Foto do fim dos anos 1870 (Wikipédia; Domínio público).}{PNLD0049-06}
PNLD0049-MP.tex-
--
PNLD0049-MP.tex:\Image{``Ressurreição de Cristo'', ícone de Andrei Rublióv (perto de 1360-1428 ou 1430) (Wikipédia; Domínio público).}{PNLD0049-13}
PNLD0049-MP.tex-
--
PNLD0049-MP.tex:\SideImage{Boris Schnaiderman (1917-2016), expoente da difusão da literatura russa no Brasil (Acervo pessoal de Gutemberg Medeiros; Direitos cedidos à Kalinka).}{PNLD0049-14}
PNLD0049-MP.tex-
--
PNLD0049-MP.tex:\Image{``No Volga'', quadro de A. Arkhípov (1862-1883) (Wikipedia; Domínio público).}{PNLD0049-07}
PNLD0049-MP.tex-
--
PNLD0049-MP.tex:\Image{Aulas de dança no Instituto Smólny (1889) (Wikipédia; Domínio público).}{PNLD0049-15}
PNLD0049-MP.tex-
--
PNLD0049-MP.tex:\Image{O autor Aleksei Tolstói (1843-1945) (Wikipédia; Domínio público).}{PNLD0049-03}
PNLD0049-MP.tex-
--
PNLD0050-MP.tex:\SideImage{Ilustração de Fido Nesti para o ``Conto do tsarévitche Cloro'', de Catarina II (Kalinka; Direitos cedidos à Kalinka).}{PNLD0050-04}
PNLD0050-MP.tex-
--
PNLD0050-MP.tex:\SideImage{Foto de Daniil Kharms (digitalizada na Casa de Púchkin, São Petersburgo. Domínio público. Direitos da foto, do acervo de Daniela Mountian, cedidos à Kalinka).}{PNLD0050-05}
PNLD0050-MP.tex-
--
PNLD0050-MP.tex:\SideImage{Autocaricatura de 1870 Edward Lear (1812-1888), pai do \emph{nonsense} (Wikipédia; Domínio público).}{PNLD0050-15}
PNLD0050-MP.tex-
PNLD0050-MP.tex:\SideImage{Retrato de Nikolai Leskóv (1831-1895) feito por Fido Nesti (Kalinka; Direitos cedidos à Kalinka).}{PNLD0050-06}
PNLD0050-MP.tex-
--
PNLD0050-MP.tex:\Image{Lev Tolstói (1828-1910) rodeado de crianças na inauguração da Biblioteca Nacional do Povo em Iásnaia Poliana. Foto (1910): A. Saviéliev (1883-1923) (Wikipédia russa; Domínio público).}{PNLD0050-07}
PNLD0050-MP.tex-
--
PNLD0050-MP.tex:\SideImage{Ilustração de Fido Nesti para ``Mumu'', conto de Ivan Turguêniev (1818-1883) (Kalinka.; Direitos cedidos à Kalinka).}{PNLD0050-08}
PNLD0050-MP.tex-
--
PNLD0050-MP.tex:\SideImage{Retrato de Anton Tchékhov (1860-1904) feito por I. Levitan (1860-1900) em 1885/1886 (Wikipedia.; Direitos cedidos à Kalinka).}{PNLD0050-03}
PNLD0050-MP.tex-
--
PNLD0050-MP.tex:\SideImage{Retrato feito por Fido Nesti de Lídia Avílova (1864-1943) (Kalinka; Direitos cedidos à Kalinka).}{PNLD0050-09}
PNLD0050-MP.tex-
--
PNLD0050-MP.tex:\SideImage{Ilustração de Fido Nesti para ``A pedrinha vermelha'', conto de Sacha Tchórny (1880-1932) (Kalinka.; Direitos cedidos à Kalinka).}{PNLD0050-10}
PNLD0050-MP.tex-
--
PNLD0050-MP.tex:\SideImage{Retrato feito por Fido Nesti de Catarina, a Grande (1729-1796) (Kalinka; Direitos cedidos à Kalinka).}{PNLD0050-11}
PNLD0050-MP.tex-
--
PNLD0050-MP.tex:\SideImage{Retrato feito por Fido Nesti de Lídia Tchárskaia (1875-1938), escritora russa infantojuvenil mais lida no início do séc. XX (Kalinka; Direitos cedidos à Kalinka).}{PNLD0050-12}
PNLD0050-MP.tex-
--
PNLD0050-MP.tex:\SideImage{Ilustração de Fido Nesti para ``O poodle branco'', conto de Aleksándr Kuprin (1870-1938) (Kalinka; Direitos cedidos à Kalinka).}{PNLD0050-13}
PNLD0050-MP.tex-
--
PNLD0050-MP.tex:\Image{Ilustração de Fido Nesti para ``O prisioneiro do Cáucaso'', conto de L. Tolstói (Kalinka; Direitos cedidos à Kalinka).}{PNLD0050-14}
PNLD0050-MP.tex-
--
PNLD0051-MP.tex:\Image{Reprodução do teto da caverna de Altamira. Divulgação. Fonte:
PNLD0051-MP.tex-(turomaquia.com/pinturas-de-altamira-arte-pre-historia).}%
--
PNLD0052-MP.tex:\SideImage{Chiquinha Gonzaga (1847--1935) aos 18 anos de idade.}{PNLD0052-03}
PNLD0052-MP.tex-
--
PNLD0052-MP.tex:\SideImage{Partitura de Odeon, de Ernesto Nazareth (Wikipedia; Domínio público)}{PNLD0052-33}
PNLD0052-MP.tex-
--
PNLD0052-MP.tex:\SideImage{Piano.}{PNLD0052-07}
PNLD0052-MP.tex-
--
PNLD0052-MP.tex:\Image{Uma família brasileira no Rio de Janeiro, 1839. (Jean-Baptiste Debret - Wilkipedia; Domínio público.)}{PNLD0052-39}
PNLD0052-MP.tex-
--
PNLD0052-MP.tex:\SideImage{Índios Tupinambás (Wikipedia/\,Jean de Léry; Domínio público)}{PNLD0052-14}
PNLD0052-MP.tex-
--
PNLD0052-MP.tex:\Image{Heitor Villa-Lobos (1887--1959)}{PNLD0052-04}
PNLD0052-MP.tex-
--
PNLD0052-MP.tex:\Image{Heitor Villa-Lobos em uma nota de cruzado, moeda brasileira dos anos 1980.}{PNLD0052-05}
PNLD0052-MP.tex-
--
PNLD0052-MP.tex:%\SideImage{}{PNLD0052-8}
PNLD0052-MP.tex-
--
PNLD0052-MP.tex:\Image{Propaganda do Estado Novo (c.\,1938) (Wikipedia; Domínio público)}{PNLD0052-35}
PNLD0052-MP.tex-
--
PNLD0052-MP.tex:\SideImage{Emicida.}{PNLD0054-21}
PNLD0052-MP.tex-
--
PNLD0052-MP.tex:\SideImage{Suposto retrato póstumo de Aleijadinho feito no século \textsc{xix} (Euclásio Ventura/ Wikpedia; Dominio público)}{PNLD0052-36}
PNLD0052-MP.tex-Falar em igrejas mineiras é falar no Aleijadinho, e vale aí procurar
--
PNLD0052-MP.tex:\SideImage{Dança de bebida dos cordados, do atlas de viagem de Spix e Martius (Wilipedia; Domínio público)}{PNLD0052-38}
PNLD0052-MP.tex-Neukomm fizeram um mapeamento e uma documentação sem paralelos da vida natural,
--
PNLD0052-MP.tex:\Image{Mapa francês da Baía de Guanabara, c. 1555 (Wikipedia; Domínio público)}{PNLD0052-15}
PNLD0052-MP.tex-
--
PNLD0052-MP.tex:\SideImage{Estátua de Carlos Gomes na Cinelândia, Rio de Janeiro (Wikipedia; Domínio púlbico)}{PNLD0052-30}
PNLD0052-MP.tex-
--
PNLD0052-MP.tex:\Image{Detalhe de Ouro Preto com a Igreja do Carmo (Wikipedia; Domínio público)}{PNLD0052-37}
PNLD0052-MP.tex-
--
PNLD0052-MP.tex:\SideImage{Desenho para a capa do libreto de  (Wikipedia; Domínio público)}{PNLD0052-31}
PNLD0052-MP.tex-
--
PNLD0052-MP.tex:\SideImage{Violino.}{PNLD0052-09}
PNLD0052-MP.tex-
--
PNLD0053-MP.tex:\Image{Depredação da sede do Sindicato dos Metalúrgicos em 1964.; Wikipedia; CC-BY-NC.}{PNLD0053-11}
PNLD0053-MP.tex-
--
PNLD0053-MP.tex:\Image{Anúncio do AI-5; Jornal ÚLtima Hora; CC-BY-NC.}{PNLD0053-04}
PNLD0053-MP.tex-
--
PNLD0053-MP.tex:\Image{Artistas protestam contra a ditadura militar em fevereiro de 1968. Na imagem, Tônia Carrero, Eva Wilma, Odete Lara, Norma Bengell e Cacilda Becker; Wikipedia; CC-BY-NC.}{PNLD0053-10}
PNLD0053-MP.tex-
--
PNLD0053-MP.tex:\Image{Manifestação das Diretas Já em Brasília, diante do Congresso Nacional.; Wikipedia; CC-BY-NC.}{PNLD0053-09}
PNLD0053-MP.tex-
--
PNLD0053-MP.tex:\SideImage{Manifestantes simulam o método de tortura conhecido como pau de arara em Brasília; Wikipedia; CC-BY-NC .}{PNLD0053-06}
PNLD0053-MP.tex-
--
PNLD0053-MP.tex:\Image{Navio Raul Soares, ca. 1926; Brasil de fato; CC-BY-NC.}{PNLD0053-05}
PNLD0053-MP.tex-
--
PNLD0053-MP.tex:\Image{Ulisses Guimarães; Wikipedia; CC-BY-NC.}{PNLD0053-08}
PNLD0053-MP.tex-
--
PNLD0053-MP.tex:\SideImage{Monumento Tortura Nunca Mais, no Recife; Wikipedia; CC-BY-NC.}{PNLD0053-07}
PNLD0053-MP.tex-
--
PNLD0054-MP.tex:\Image{Sidney Rocha (Arquivo do autor)}{PNLD0054-03.png}
PNLD0054-MP.tex-
--
PNLD0054-MP.tex:  \SideImage{Alphonsus de Guimarães (1870--1921) (Desenho de Calixto; )}{PNLD0054-20} %Sofia: não encontrei o crédito da imagem abaixo
PNLD0054-MP.tex-
--
PNLD0054-MP.tex:\SideImage{Emicida (1985-) (Daryan Dornelles; Divulgação)}{PNLD0054-21}
PNLD0054-MP.tex-
--
PNLD0054-MP.tex:\Image{Claude Debussy (1862--1918) (J. Cuthbert Hadden; Domínio Público)}{PNLD0054-22}
PNLD0054-MP.tex-\marginnote{\footnotesize \tt\textbf{éc.fra.se} s.m.\\ a palavra ecfrasis ou écfrase vem do grego 
--
PNLD0054-MP.tex:\SideImage{Stéphane Mallarmé (1842--1898) (Paul Nadar; Domínio Público)}{PNLD0054-23}
PNLD0054-MP.tex-
--
PNLD0054-MP.tex:\Image{Oficina de dança contemporânea (Mallu Silva/Labfoto; CC-BY 2.0)}{PNLD0054-05.png}
PNLD0054-MP.tex-
--
PNLD0054-MP.tex:% \Image{Ana Botafogo é atriz e uma das mais conhecidas bailarinas brasileiras. (Roberto Filho; CC-BY 2.1)}{PNLD0054-08.png}
PNLD0054-MP.tex-
--
PNLD0054-MP.tex:\Image{Pina Bausch foi uma importante dançarina e coreógrafa alemã. (Raphael Labbé; CC-BY-SA 3.0)}{PNLD0054-06.png}
PNLD0054-MP.tex-
--
PNLD0054-MP.tex:\Image{Quadro "O Nascimento de Vênus", de Sandro Botticceli representa a deusa do amor, cujo nome grego é Afrodite. (Sandro Botticelli; Domínio Público)}{PNLD0054-09.png}
PNLD0054-MP.tex-
--
PNLD0054-MP.tex:\SideImage{Estátua de Afrodite exposta no Museu Arqueológico de Nápoles. 
PNLD0054-MP.tex-É uma cópia da estátua grega de 310-200 A.C. 
--
PNLD0054-MP.tex:\SideImage{Estátua de Afrodite exposta na "Biblioteca Nazionale Marciana", em Veneza (Biblioteca Nazionale Marciana; CC-BY 3.0)}{PNLD0054-10.png}
PNLD0054-MP.tex-
--
PNLD0054-MP.tex:\Image{Cecília Kerche é uma famosa bailarina brasileira, que assumiu em 1986 a posição de primeira bailarina do Teatro Municipal do Rio de Janeiro. (Fernando Frazão/Agência Brasil; CC-BY 2.0)}{PNLD0054-07.png}
PNLD0054-MP.tex-
--
PNLD0054-MP.tex:\SideImage{Espectáculo da Companhia de Dança Deborah Colker (Rosano Mauro; CC-BY 2.0)}{PNLD0054-04.png}
PNLD0054-MP.tex-
--
PNLD0059-MP.tex:\SideImage{Frontispício do manuscrito de Ximénez em que se lê <<[aqui] começam as histórias da origem dos índios desta província da Guatemala. Traduzido da língua Quiché para o castelhano, para conveniência dos ministros do Santo Evangelho, pelo R[everend] P[adre] F[riar] Francisco Ximénez, sacerdote doutrinal do conselho real de Santo Tomás Chilá.>> (Ohio State University; Domínio Público)}{PNLD0059-21}
PNLD0059-MP.tex-
--
PNLD0059-MP.tex:\Image{Fragmento de tecido pré-colombiano pertencente à coleção Landman em posse do MASP-SP, que foi exposta ao público em 2019. Divulgação.}{PNLD0059-22}
PNLD0059-MP.tex-
--
PNLD0059-MP.tex:\Image{Fragmento de tecido pré"-colombiano exposto no MASP-SP. Divulgação}{PNLD0059-24}
PNLD0059-MP.tex-
--
PNLD0059-MP.tex:\Image{Manuscrito de \textit{Popol Vuh}, compilado pelo frei dominicano Francisco Ximénez.(Francisco Ximénez; Domínio Público)}{PNLD0059-03.png}
PNLD0059-MP.tex- 
--
PNLD0059-MP.tex:\Image{Exemplo de uma data do sistema de contagem de tempo dos maias pré-hispânicos conhecido como Conta Longa. (Francisco França; Domínio Público)}{PNLD0059-08.png}
PNLD0059-MP.tex-
--
PNLD0059-MP.tex:\Image{Cerimônia para a comemoração de um novo período na contagem do tempo dos antigos maias. (Francisco França; Domínio Público)}{PNLD0059-06.png}
PNLD0059-MP.tex-
--
PNLD0059-MP.tex:\Image{Deus do milho emergindo da carapaça de uma tartaruga. (Francisco França; Domínio Público)}{PNLD0059-10.png}
PNLD0059-MP.tex-
--
PNLD0059-MP.tex:\Image{Huracán, o Coração do Céu. (Francisco França; Domínio Público)}{PNLD0059-05.png}
PNLD0059-MP.tex-
--
PNLD0059-MP.tex:\Image{Gêmeos enfrentam Vucub"-Caquix (desenho da imagem presente no Prato Blom). (Francisco França; Domínio Público)}{PNLD0059-04.png}
PNLD0059-MP.tex-
--
PNLD0059-MP.tex:\Image{Cena de uma prática de autossacrifício, com personagem à direita, extraindo sangue da língua. (Francisco França; Domínio Público)}{PNLD0059-09.png}
PNLD0059-MP.tex-
--
PNLD0060-MP.tex:\SideImage{Reprodução do livro de Robert Taylor Pritchett da primeira edição ilustrada de \textit{Murray}, 1890: Beagle no Estreito de Magalhães no Monte Sarmiento no Chile. CC-BY.}{PNLD0060-21}
PNLD0060-MP.tex-
PNLD0060-MP.tex:\Image{Trajeto da viagem da expedição Beagle.}{PNLD0060-20}
PNLD0060-MP.tex-
--
PNLD0060-MP.tex:\Image{Desenho botânico da flor prímula (Alex Cerveny; Direitos cedidos à Ubu)}{PNLD0060-08.png}
PNLD0060-MP.tex-
--
PNLD0060-MP.tex:\Image{Desenho de uma mariposa (Alex Cerveny;Direitos cedidos à Ubu}{PNLD0060-09.png}
PNLD0060-MP.tex-
--
PNLD0060-MP.tex:\Image{Desenho de um pássaro quiuí (Alex Cerveny; Direitos cedidos à Ubu)}{PNLD0060-10.png}
PNLD0060-MP.tex-
--
PNLD0060-MP.tex:\Image{Ilustração da viagem marítima. (Alex Cerveny; Direitos cedidos à Ubu)}{PNLD0060-05.png}
PNLD0060-MP.tex-
--
PNLD0060-MP.tex:\Image{Estrutura do navio Beagle (Alex Cerveny; Direitos cedidos à Ubu)}{PNLD0060-04.png}
PNLD0060-MP.tex-
--
PNLD0060-MP.tex:\Image{Diagrama de ramificação dos roedores e marsupiais a partir de uma mesma
PNLD0060-MP.tex-matriz. Década de 1850. (Charles Darwin; Domínio Público)}{PNLD0060-07.png}
--
PNLD0060-MP.tex:\Image{Esquema geral de ramificação das espécies por seleção natural. Década
PNLD0060-MP.tex-de 1840. (Charles Darwin; Domínio Público)}{PNLD0060-06.png}
--
PNLD0061-MP.tex:\Image{Liberato de Castro Carreira.~Descripção da epidemia da febre amarella
PNLD0061-MP.tex-na Provincia do Ceará em 1851 e 1852.~Rio de Janeiro, RJ: Typ. de N.L.
--
PNLD0061-MP.tex:\Image{Marc Ferrez. {[}Laboratório{]}. {[}S.l.: s.n.{]}, {[}1886-1889{]}. 19 x
PNLD0061-MP.tex-25cm em c. 31,5 x 47,5. Disponível em:
--
PNLD0061-MP.tex:\Image{Rue de Ouvidor\textbf{.~}{[}S.l.: s.n.{]}, c. 1890. 1 foto, gelatina e
PNLD0061-MP.tex-prata, p\&b, 36,2 x 30cm. Biblioteca nacional}{PNLD0061-03}
--
PNLD0061-MP.tex:\Image{Marc Ferrez, ~{[}Avenida Central --- Vista para o Norte{]}.~Rio de
PNLD0061-MP.tex-Janeiro, RJ: {[}s.n.{]}, 15 nov. 1906. 1 reprod. fotom., gelatina, p\&b,
--
PNLD0061-MP.tex:\SideImage{\textit{Sair de grande noite: Ensaio sobre a África
PNLD0061-MP.tex-descolonizada}, por A. Mbembe}{PNLD0061-08}
--
PNLD0062-MP.tex:\Image{O autor Igor Mendes (Bruna Freire; Acervo pessoal do autor)}{PNLD0062-03.jpg}
PNLD0062-MP.tex-
--
PNLD0062-MP.tex:\Image{Advertência (Isabel Teixeira (Ateliê Fora de Esquadro); Direitos cedidos à n"-1)}{PNLD0062-06.png}
PNLD0062-MP.tex-
--
PNLD0062-MP.tex:\Image{Ilustração a partir de foto de Bruna Freire (Isabel Teixeira (Ateliê Fora de Esquadro); Direitos cedidos à n"-1)}{PNLD0062-07.png}
PNLD0062-MP.tex-
--
PNLD0062-MP.tex:\Image{Ilustração penitenciária (Isabel Teixeira (Ateliê Fora de Esquadro); Direitos cedidos à n"-1)}{PNLD0062-04.png}
PNLD0062-MP.tex-
--
PNLD0062-MP.tex: \Image{Ilustração passagem (Isabel Teixeira (Ateliê Fora de Esquadro); Direitos cedidos à n"-1)}{PNLD0062-05.png}
PNLD0062-MP.tex-
--
PNLD0062-MP.tex:\Image{Identidade (Isabel Teixeira (Ateliê Fora de Esquadro); Direitos cedidos à n"-1)}{PNLD0062-08.png}
PNLD0062-MP.tex-
--
PNLD0062-MP.tex:\Image{Ilustração cela (Isabel Teixeira (Ateliê Fora de Esquadro); Direitos cedidos à n"-1)}{PNLD0062-09.png}
PNLD0062-MP.tex-
--
PNLD0062-MP.tex:\Image{Ilustração do livro a partir de foto de Bruna Freire  (Isabel Teixeira (Ateliê Fora de Esquadro); Direitos cedidos à n"-1)}{PNLD0062-10.png}
PNLD0062-MP.tex-
--
PNLD0062-MP.tex:\Image{Ilustração brasão penitenciária (Isabel Teixeira (Ateliê Fora de Esquadro); Direitos cedidos à n"-1)}{PNLD0062-11.png}
PNLD0062-MP.tex-
--
PNLD0062-MP.tex:\Image{Ilustração relatos  (Isabel Teixeira (Ateliê Fora de Esquadro); Direitos cedidos à n"-1)}{PNLD0062-12.png}
PNLD0062-MP.tex-
--
PNLD0062-MP.tex:\Image{``Em defesa dos presos e perseguidos políticos'' (Isabel Teixeira (Ateliê Fora de Esquadro); Direitos cedidos à n"-1)}{PNLD0062-13.png}
PNLD0062-MP.tex-
--
PNLD0062-MP.tex:\Image{Luta--luto (Isabel Teixeira (Ateliê Fora de Esquadro); Direitos cedidos à n"-1)}{PNLD0062-16.png}
PNLD0062-MP.tex-
--
PNLD0062-MP.tex:\Image{Carta para Igor Mendes (Isabel Teixeira (Ateliê Fora de Esquadro); Direitos cedidos à n"-1)}{PNLD0062-18.png}
PNLD0062-MP.tex-
